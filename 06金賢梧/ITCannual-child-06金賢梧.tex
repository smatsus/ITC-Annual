\subsection{研究報告(金 賢梧)}

セキュリティ応用のための高速ビジョンネットワークシステムの研究

【背景】分散型高速カメラネットワークを用いた画像処理アプリケーションでは、各カメラにおける同期撮像・画像処理が重要な場合が少なくない。実際、複数のカメラビューに渡る対象物体の位置検出や高速移動体の3次元形状復元などにおいて、画像データの時刻同期精度は計測精度に大きく影響する。一般的な撮像速度である30fpsのカメラを用いる時と比べて、1,000fpsの高速撮像が可能な高速カメラでは画像フレーム間の時間間隔が非常に短いため、注意深く撮像タイミングを制御する必要がある。さらに、実時間フィードバックシステムとの連動のためには、同時撮像後の画像データからの処理情報が同じタイミングで統合・共有される必要もある。多数の高速カメラから構成される分散型ネットワークシステムにおいて同期撮像・同期データ処理を保証するためには、高速なデータ処理方法を工夫した効率的なシステム構造の設計が重要である。

【研究内容と成果】本研究では、複数の高速カメラによる最大1,000fpsの高速撮像および画像データ処理を同期して行うことができる分散型高速カメラネットワークを構築して、同期精度の評価を行った。そのため、階層的並列分散構造を持つネットワーク構造の下、Message Passing Interface (MPI)を導入することで、Reference BroadcastおよびPrecision Time Protocol (PTP)による撮像制御と同期データ転送を高速な撮像速度に合わせて実現することができた。また、ネットワーク上の端末となる高速カメラを画像処理やデータ転送機能までを持つスマートカメラとすることで、ネットワークにおいてはデータ量の小さい、必要な情報だけが転送されるようになり、現在16台規模のネットワークシステムにおいての撮像同期の偏差が最大数十μsと1,000fpsの高速撮像・データ処理に向けて十分な同期精度が達成できた。 本分散型高速カメラネットワークの応用先としては、セキュリティー分野のモニターリングシステムや高速移動体の計測システムおよび高速視覚基盤動作入力装置などが考えられる。
