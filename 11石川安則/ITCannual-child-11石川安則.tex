\subsection{研究報告(石川 安則)}

 NEDOプロジェクト「高度なIoT社会を実現する横断的技術開発/高速ビジョンセンサネットワークによる実時間IoTシステムと応用技術開発」テーマにおける、実施項目1「センサネットワーク構造および全体システムのアーキテクチャの提案」に関する研究を行った。全般的な目標としては、高速センサネットワーク構造および全体システムアーキテクチャの構築に向けて、FAシステム等における産業用ロボットや、自動検査装置などにおけるIoTネットワークの統合的なアーキテクチャ構成に関して調査・検討を行い、これにより全体システムアーキテクチャとの整合性を評価するとともに、高速センサネットワーク構造の構築に適用可能なシステムアーキテクチャのフレームワークについて提案を行う。より具体的には、新しい実時間IoTシステムの中で、センサフュージョン技術をベースとして、センサデータとこれをコントロールするセンサネットワーク構造を構築し、高速ビジョンを含むセンサネットワークシステムにおいてサブミリ秒の同期精度を実現するシステムの提案と、1kHzのフィードバック系を目標とした評価システムを構築し、最終評価を行う。複数ビジョンや多種センサを用いたセンサネットワークシステムの構築に関しては、高速ビジョンを含むセンサネットワークシステムにおいてサブミリ秒の同期精度を実現するため、PTPと呼ばれる精密時刻同期プロトコル(Precision Time Protocol、IEEE1588)を導入し、これにより各センサノードの時間同期を実現した。さらに、ネットワークノード数を拡大した場合(1000ノード程度)のパフォーマンス測定の検討を行い、超小型のボードコンピュータ(ラズベリーパイの廉価版)を用いて、PTPデーモンを実装してソフトウェアタイムスタンプによる同期測定が可能であることを確認した。また、さらに小型のマイコンボード(WiFiやBluetoothをオンチップで搭載している安価なESP32マイコンボード)を用いた場合の検証実験について検討を行った。100台程度のノード数でWiFi接続によるネットワークを構築し、これを1ブロックとして上位の有線ルータの下に複数のブロックを接続し、階層的構造とするアーキテクチャにより、これを10ブロック接続することで合計1000台のノード接続でパフォーマンス測定が可能となる。
