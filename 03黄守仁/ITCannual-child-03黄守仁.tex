\subsection{研究報告(黄 守仁)}
 本年度は人間機械協調、ロボットの知能化を目指した動的補償ロボット、高周波外部フィードバックに基づく電気刺激など研究課題を巡って研究を行った。人間機械協調に関しては、人間の認知能力と機械(ロボット)の高速・高精度な動作を相互補完的に組み合わせることを目指して、これまでに構築した視覚・触覚などを用いた感覚提示による人間機械協調システムに基づいて、力覚提示によるヒューマンロボットインタラクションと人間の両腕同期運動現象(例えば、左右の手(腕)で異なった運動を同時に行おうとしても、両手が同じような動きになる傾向、脳にとって最も基本的な仕組みとも考えられる)を統合する研究を行った。次に、3自由度動的補償モジュールを新規開発し、高速高精度塗布や溶接など応用に向けた知能産業用ロボットに関する研究も推進した。また、高周波電気刺激装置を開発し、高周波外部フィードバック情報による人間の上腕に対する電気刺激制御の基礎実験環境を構築した。これら研究課題の実施により、特許出願、著書、国際学会誌論文、国際学会、国内学会など含む合計5件の研究成果が得られた。競争的資金の獲得に関しては、若手研究(研究課題「Control of human upper limb by electrical stimulation for accurate motion with external mechanical assistance of high bandwidth: basic mechanism and modeling」、研究代表者)、基盤研究(S)(研究課題「超高速ビジョン・トラッキング技術を用いた次世代情報環境システムの創生」、研究分担者)などが採択された。