\subsection{研究報告(胡 云普)}

Doppler Time-of-Flight Imagingを用いたあらゆる環境に向ける運動計測

1.背景 本研究では、三次元速度に繋がるドップラー効果をTime-of-Flightカメラで計測することで、あらゆる環境に向ける運動計測を実現する。既存手法が抱える問題の一つは、常に計測対象に様々な仮定を必要とする点である。例えば、対象の表面に細かいテクスチャが必要であることや、対象に事前にセンサを設置できる必要があることなどが挙げられる。これらの仮定は、実際の応用において適用できない場合が多い。これに伴い、同仮定を必要とする手法も利用が困難である。故に、汎用的に利用可能な三次元運動解析の実現に向けて、対象を限定しないことや非侵襲であることが求められている。この問題を解決するため、高速ビジョンで対象物の三次元速度をドップラー効果から直接に推定するのは、本研究の着目点となります。

2.内容 本年度の研究は、ドップラー周波数を取得できるハードウェアプラットフォームの構築、及び高精度の距離速度推定ためのデータ処理アルゴリズムを対象とする。具体的に、まず、ドップラー周波数を取得するため、ヘテロダインモードで撮像可能なシステムを構築必要がある。また、関連研究ではヘテロダインTime-of-Flightについて検討は不十分だと考えられて、システムのパラメーターおよび作動方式についてより論理的な分析が必要とする。最後に、高精度で距離と速度情報を抽出するため、関連研究では直接法を利用したが、この手法はノイズの影響を受けやすい問題があります。本研究では、高精度、ノイズロバストな速度推定アルゴリズムを提案する。

3.具体的成果 まず、ヘテロダインモードで撮像可能なTime-of-Flightカメラシステムを構築した。本システムは従来のTime-of-Flightシステムと区別して、照明と調製に異なる周波数を利用することで、環境中の位相とドップラー周波数を取得する。このようなヘテロダインモードTime-of-Flight撮像を位相ー周波数の二次元相関関数としてを分析し、同システムの最適な作動パラメーターを提案した。三次元距離と速度の推定手法として、直接法と二次元相関図での最適化手法、二つを提案した。二つの手法とも、従来手法と比べて、明らかに高い計測精度を実現した。

