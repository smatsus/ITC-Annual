\begin{招待講演}{1}

\bibitem{05宮下令央01}
宮下令央: 世界を書き換えるダイナミックプロジェクションマッピング, IPSJ ONE 2021,2021/3.

\bibitem{05宮下令央02}
宮下令央: 目に見えるものは真実か, 金沢大学人間社会学域学校教育学類附属高等学校, 同窓生による特別授業,2020/12.

\bibitem{07平野正浩01}
岸則政, 妹尾拓, 平野正浩:高速画像処理システムの自動運転への役割—素早い危険予知—,第11回横貫連合コンファレンス,2020.


\end{招待講演}

\begin{招待論文}{1}

\bibitem{02早川智彦01}
早川智彦,久保田祐貴,望戸雄史,柯毓珊,石川正俊:モーションブラー補償による高速撮像技術のインフラ検査への応用, 光学, 50巻, 2号, pp. 61-67,2021


\end{招待論文}

\begin{受賞}{1}

\bibitem{01石川正俊01}
小山佳祐,下条誠,妹尾拓,石川正俊: 小型・低摩擦アクチュエータ"MagLinkage"を用いた低衝撃・ノンストップ把持, 第37回日本ロボット学会学術講演会 (RSJ2019)/予稿集, 3E2-07,令和2年度日本ロボット学会優秀研究・技術賞,2020/10.

\bibitem{01石川正俊02}
小山佳祐, 下条誠, 妹尾拓, 石川正俊: 小型・低摩擦アクチュエータMagLinkageの開発とハンド応用, 日本機械学会ロボティクス・メカトロニクス講演会2019(ROBOMECH2019)/講演論文集, 2P1-H02,令和元年度日本機械学会ロボティクス・メカトロニクス部門ROBOMEC表彰,2020/05.

\bibitem{02早川智彦02}
東京大学(早川智彦, 望戸雄史, 栃岡陽麻里, 石川正俊),中日本高速道路株式会社(亀岡弘之, 藤田友一郎, 大西偉允),高速道路のトンネルにおける時速100km走行での覆工コンクリート高解像度変状検出手法,第4回インフラメンテナンス大賞/国土交通大臣賞,2021/01.

\bibitem{02早川智彦03}
早川智彦,柯毓珊,望戸雄史,石川正俊:モーションブラー補償撮像手法を利用した走行型点検車両の照明要件―高速道路のトンネル覆工表,第42回照明学会東京支部大会,最優秀研究発表者賞,予稿集,pp. B-6:1-B-6:2,2020/12.

\bibitem{02早川智彦04}
門脇 拓也, 丸山 三智佳, 早川 智彦, 松澤 直熙, 岩崎 健一郎, 石川 正俊: 身体感覚と視覚情報にずれが生じる没入環境における低遅延な映像のユーザーへの影響,日本バーチャルリアリティ学会論文誌, 24巻, 1号, pp.23-30,第22回日本VR学会論文賞,2019.

\bibitem{02早川智彦05}
大学 情報基盤センター 石川・早川・黄・末石・宮下研究室:特定速度で移動している人だけに伝達可能な二次元情報提示システム(Bilateral Motion Display),デジタルコンテンツ協会 Innovative Technologies 2020,スポンサー賞,2020/11.

\bibitem{02早川智彦06}
大学 情報基盤センター 石川・早川・黄・末石・宮下研究室:特定速度で移動している人だけに伝達可能な二次元情報提示システム(Bilateral Motion Display),デジタルコンテンツ協会 Innovative Technologies 2020,Special Prize - Vision -,2020/11.

\bibitem{02早川智彦07}
東京大学・中日本高速道路株式会社:高速道路のトンネル覆工コンクリートにおける時速100km走行での4K高解像度変状検出システム,第9回ロボット大賞 優秀賞(研究開発部門),2021/03.

\bibitem{04末石智大01}
末石智大,西薗良太,石川正俊:ベクター型レーザー投影系におけるM系列破線マーカーを用いたロバスト高速自己姿勢推定,第21回計測自動制御学会システムインテグレーション部門講演会 (SI2020),講演会論文集, pp.1445-1448,優秀講演賞,2020/12.

\bibitem{04末石智大02}
松本明弓,末石智大,石川正俊:注視点追従高解像度投影に向けた高速視線推定システム, 第21回計測自動制御学会システムインテグレーション部門講演会 (SI2020),講演会論文集, pp.1886-1889,優秀講演賞,2020/12.

\bibitem{07平野正浩02}
川原 大宙,妹尾 拓,石井 抱,平野 正浩,岸 則政,石川 正俊:第21回計測自動制御学会システムインテグレーション部門講演会(SI2020)優秀講演賞,2020/12.

\bibitem{09角博文01}
上村将之・王 澤・竹原浩成(奈良先端大)・角 博文(東大)・田代洋行・春田牧人・笹川清隆・太田 淳(奈良先端大):近赤外カラー高速眼底カメラ向けイメージセンサへのモザイク多層膜干渉フィルタ搭載と評価,情報センシング研究会(IST)高機能イメージセンシングとその応用, 優秀発表賞,20/07.

\bibitem{SatoruNakamura201} 
小風尚樹, 中村覚, 永崎研宣:
 情報処理学会 人文科学とコンピュータシンポジウム「じんもんこん2019」 学生奨励賞 構造化記述された財務記録史料データの分析手法の開発:イギリスの船舶解体業を事例に
\end{受賞}

\begin{著書}{1}

\bibitem{03黄守仁01}
Shouren Huang, Yuji Yamakawa and Masatoshi Ishikawa:Dynamic Compensation Framework to Improve the Autonomy of Industrial Robots. IntechOpen,Industrial Robotics-New Paradigms,2020/09.

\bibitem{SatoruNakamura301} 
中村覚(担当:共編者(共編著者)):
 デジタルアーカイブ・ベーシックス 2:災害記録を未来に活かす, 2019年8月 (ISBN: 9784585202820)
\end{著書}

\begin{雑誌論文}{1}

\bibitem{01石川正俊03}
Yuji Yamakawa, Yutaro Matsui and Masatoshi Ishikawa: Development of a Real-Time Human-Robot Collaborative System Based on 1 kHz Visual Feedback Control and Its Application to a Peg-in-Hole Task, Sensors, Vol.21, Issue 2, Article No. 663 ,2021/1.

\bibitem{01石川正俊04}
Zhangxu Pan, Chan Guo, Xianchi Wang, Jiucheng Liu, Ruimin Cao, Yanfen Gong, Jiantai Wang, Ningyang Liu, Zhitao Chen, Lihui Wang, Masatoshi Ishikawa, and Zheng Gong: Wafer-Scale Micro-LEDs Transferred onto an Adhesive Film for Planar and Flexible Displays, Advanced Materials Technologies, 2000549, pp.1-11,2020/8.

\bibitem{01石川正俊05}
Kenichi Murakami, Koki Ishimoto, Taku Senoo and Masatoshi Ishikawa: Human Robot Hand Interaction with Plastic Deformation Control, Robotics, Vol.9, No.3, Article No.73,	2020/9.

\bibitem{02早川智彦08}
Y. Kubota, T. Hayakawa, and M. Ishikawa: Dynamic perceptive compensation for the rotating snakes illusion with eye tracking, PLoS ONE 16(3), 2021.

\bibitem{03黄守仁02}
Shouren Huang, Masatoshi Ishikawa, Yuji Yamakawa:A coarse-to-fine framework for accurate positioning under uncertainties—from autonomous robot to human–robot system,Int J Adv Manuf Technol,2020.

\bibitem{05宮下令央03}
Mikihiro Ikura, Leo Miyashita, Masatoshi Ishikawa: Stabilization System for UAV Landing on Rough Ground byAdaptive 3D Sensing and High-speed Landing Gear Adjustment, Journal of Robotics and Mechatronics, Vol.33, No.1 (2021),2021/2.

\bibitem{12李ソ賢01}
Seohyun Lee, Hyuno Kim, Hideo Higuchi, and Masatoshi Ishikawa: Visualization Method for the Cell-level Vesicle Transport Using Optical Flow and Diverging Colormap, Sensors, Vol. 21(2), No. 522, pp. 1-13,2021.

\bibitem{JIANG1901}
Zipei Fan, Xuan Song, Renhe Jiang, Quanjun Chen, and Ryosuke Shibasaki:
Decentralized Attention-based Personalized Human Mobility Prediction, Proceedings of the ACM on Interactive Mobile Wearable and Ubiquitous Technologies, Vol.3, No.4, pp1-26, December 2019.
\bibitem{JIANG2001}
Zipei Fan, Xuan Song, Quanjun Chen, Renhe Jiang, Ryosuke Shibasaki, and Kota Tsubouchi:  
Trajectory fingerprint: one-shot human trajectory identification using Siamese network, CCF Transactions on Pervasive Computing and Interaction, 2(2), 113-125, 2020.
\bibitem{JIANG2002}
Renhe Jiang, Quanjun Chen, Zekun Cai, Zipei Fan, Xuan Song, Kota Tsubouchi, and Ryosuke Shibasaki: 
Will You Go Where You Search? A Deep Learning Framework for Estimating User Search-and-Go Behavior, Neurocomputing, 2020.
\bibitem{JIANG2003}
Renhe Jiang, Xuan Song, Zipei Fan, Tianqi Xia, Zhaonan Wang, Quanjun Chen, Zekun Cai, and Ryosuke Shibasaki: 
Transfer Urban Human Mobility via POI Embedding over Multiple Cities, ACM/IMS Trans. Data Sci. 2, 1, Article 4, 26 pages, January 2021.
\bibitem{LMY01}
T. Lee, S. Matsushima, K. Yamanishi: “Grafting for combinatorial binary model using frequent itemset mining,” Data Mining and Knowledge Discovery, 34(1), pp. 101-123 (2020)
\bibitem{FMY01}
Y. Fu, S. Matsushima, K. Yamanishi: “Model Selection for Non-Negative Tensor Factorization with Minimum Description Length,” Entropy 2019, 21, 632.
	
\end{雑誌論文}

\begin{査読付}{1}

\bibitem{01石川正俊06}
Satoshi Tanaka, Keisuke Koyama, Taku Senoo, Makoto Shimojo, and Masatoshi Ishikawa: High-speed Hitting Grasping with Magripper, a Highly Backdrivable Gripper using Magnetic Gear and Plastic Deformation Control, 2020 IEEE/RSJ International Conference on Intelligent Robots and Systems (IROS2020), Proceedings, pp. 9137 - 9143,2020/10.

\bibitem{01石川正俊07}
Ryosuke Higo, Taku Senoo and Masatoshi Ishikawa:Dynamic In-Hand Regrasping Using a High-Speed Robot Hand and High-Speed Vision, 1st Virtual IFAC World Congress (IFAC-V 2020) (Virtual Conference),pp.985:1-985:6,2020/7.

\bibitem{01石川正俊08}
Fumiya Shimada, Kenichi Murakami, Taku Senoo and Masatoshi Ishikawa:Bolt loosening detection using multi-purpose robot hand, 2020 IEEE/ASME International Conference on Advanced Intelligent Mechatronics (Virtual Conference,2020.7.9), pp.1860-1866,2020/07.

\bibitem{01石川正俊09}
Kenichi Murakami, Koki Ishimoto, Taku Senoo and Masatoshi Ishikawa:Robot Hand Interaction Using Plastic Deformation Control with Inner Position Loop, 2020 IEEE/ASME International Conference on Advanced Intelligent Mechatronics (Virtual Conference),pp.1748-1753,2020/07.

\bibitem{01石川正俊10}
Satoshi Tanaka, Keisuke Koyama, Taku Senoo, and Masatoshi Ishikawa: Adaptive Visual Shock Absorber with Visual-based Maxwell Model Using Magnetic Gear, The 2020 International Conference on Robotics and Automation (ICRA2020) (Paris), Proceedings, pp. 6163-6168,2020/06.

\bibitem{02早川智彦09}
Tomohiko Hayakawa, Haruka Nakane and Masatoshi Ishikawa:Motion-blur Compensation System Using a Rotated Acrylic Cube with Visual Feedback, 1st Virtual IFAC World Congress (IFAC-V 2020), pp. 696:1-696:4,2020/07.

\bibitem{02早川智彦10}
Yuriko Ezaki, Yushi Moko, Haruka Ikeda, Tomohiko Hayakawa and Masatoshi Ishikawa:Extension of the Capture Range Under High-Speed Motion Using Galvanometer Mirror, 2020 IEEE/ASME International Conference on Advanced Intelligent Mechatronics (Virtual Conference), pp.1854-1859,2020/07.

\bibitem{02早川智彦11}
Y. Kubota, T. Hayakawa, M. Ishikawa: Quantitative Perception Measurement of the Rotating Snakes Illusion Considering Temporal Dependence and Gaze Information, Symposium on Eye Tracking Research and Applications (ETRA '20 Short Papers) (online),Proceedings, No.45, pp. 1-4,2020/05.

\bibitem{02早川智彦12}
Y. Kubota, T. Hayakawa, Y. Ke, Y. Moko, M. Ishikawa: High-speed motion blur compensation system in infrared region using galvanometer mirror and thermography camera, SPIE Sensors and Smart Structures Technologies for Civil, Mechanical and Aerospace Systems 2020,Proceedings,1137919,2020/04.

\bibitem{02早川智彦13}
池田遼,早川智彦,栃岡陽麻里,石川正俊: 観測者の視線運動に応じた残像効果による指向性ディスプレイ, インタラクション2021論文集(オンライン),pp.57-63,2020/04.

\bibitem{03黄守仁03}
Shouren Huang, Keisuke Koyama, Masatoshi Ishikawa, and Yuji Yamakawa:Human-Robot Collaboration with Force Feedback Utilizing Bimanual Coordination,In Companion of the 2021 ACM/IEEE International Conference on Human-Robot Interaction (HRI ’21 Companion),2021

\bibitem{04末石智大03}
Tomohiro Sueishi, Arata Jingu, Shoji Yachida, Michiaki Inoue, Yuka Ogino, and Masatoshi Ishikawa: Dynamic Iris Authentication by High-Speed Gaze and Focus Control, 2021 IEEE/SICE International Symposium on System Integration (SII2021) ,Proceedings, pp.813-814,2021/1.

\bibitem{04末石智大04}
Ryota Nishizono, Tomohiro Sueishi, and Masatoshi Ishikawa: EmnDash,M-sequence Dashed Markers on Vector-based Laser Projection for Robust High-speed Spatial Tracking, IEEE International Symposium on Mixed and Augmented Reality Adjunct (ISMAR-Adjunct2020), pp. 195-200,2020/11.

\bibitem{04末石智大05}
Murtuza Petladwala, Tomohiro Sueishi, Shoji Yachida, and Masatoshi Ishikawa: High-Speed Occlusion Recovery Method for Multiple Fish Visual Tracking, The 42nd Annual International Conference of the IEEE Engineering in Medicine and Biology Society (EMBC2020),Proceedings, MoAT14.12,2020/7.

\bibitem{04末石智大06}
Yuri Mikawa, Tomohiro Sueishi, Yoshihiro Watanabe, and Masatoshi Ishikawa: Projection mapping system to a widely dynamic sphere with circumferential markers, IEEE International Conference on Multimedia and Expo (ICME2020), pp. 1-6,2020/7.

\bibitem{05宮下令央04}
Kentaro Fukamizu, Leo Miyashita, Masatoshi Ishikawa: ElaMorph Projection,Deformation of 3D Shape by Dynamic Projection Mapping, International Symposium on Mixed and Augmented Reality (ISMAR2020), Recife, Brazil (Virtual conference), pp.220-229, 9-13 ,2020/11.

\bibitem{05宮下令央05}
Leo Miyashita, Masatoshi Ishikawa: Wearable DPM System with Intelligent Imager and GPU, International Conference on Artificial Intelligence Circuits and Systems (AICAS2020), Live Demos, Proceedings pp.129-130, Genoa, Italy (Virtual Conference),2020/9.

\bibitem{07平野正浩03}
Kento Yabuuchi, Masahiro Hirano, Taku Senoo, Norimasa Kishi and Masatoshi Ishikawa: Real-Time Traffic Light Detection with Frequency Patterns Using a High-Speed Camera, Sensors, Vol.20, No.14, Article No.4035, pp.1-18,2020/07.

\bibitem{08田畑智志01}
Lihui Wang, Hongjin Xu, Satoshi Tabata, Yunpu Hu, Yoshihiro Watanabe, and Masatoshi Ishikawa:High-Speed Focal Tracking Projection Based on Liquid Lens, ACM SIGGRAPH 2020 Emerging Technologies (SIGGRAPH '20),2020/08.

\bibitem{12李ソ賢02}
Seohyun Lee, Hyuno Kim, Hideo Higuchi, and Masatoshi Ishikawa: A machine learning approach to transport categorization for vesicle tracking data analysis, SPIE Photonics West BiOS 2021 (as Online conference) Proceedings, pp. 1-5,2021/3.

\bibitem{12李ソ賢03}
Seohyun Lee, Hyuno Kim, Hideo Higuchi, and Masatoshi Ishikawa: Estimation of vesicle transport near the cellular membrane using image processing, 2020 OSA Imaging and Applied Optics Congress / Proceedings, JF4E.2,2020/06.

\bibitem{JIANG1902}
Renhe Jiang, Xuan Song, Dou Huang, Xiaoya Song, Tianqi Xia, Zekun Cai, Zhaonan Wang, Kyoung-Sook Kim, and Ryosuke Shibasaki:
Deepurbanevent: A system for predicting citywide crowd dynamics at big events, Proceedings of The 25th ACM SIGKDD International Conference on Knowledge Discovery \& Data Mining (KDD'19), pp2114-2122, July 2019.
\bibitem{JIANG1903}
Zipei Fan, Quanjun Chen, Renhe Jiang, Ryosuke Shibasaki, Xuan Song, and Kota Tsubouchi:
Deep Multiple Instance Learning for Human Trajectory Identification, Proceedings of the 27th ACM SIGSPATIAL International Conference on Advances in Geographic Information Systems (SIGSPATIAL'19), pp512-515, November 2019.
\bibitem{JIANG1904}
Xiaodan Shi, Xiaowei Shao, Zipei Fan, Renhe Jiang, Haoran Zhang, Zhiling Guo, Guangming Wu, Wei Yuan, and Ryosuke Shibasaki:
Multimodal Interaction-Aware Trajectory Prediction in Crowded Space, Proceedings of The Thirty-Fourth AAAI Conference on Artificial Intelligence (AAAI'20), pp11982-11989, February 2020.
\bibitem{JIANG2004}
Satoshi Miyazawa, Xuan Song, Renhe Jiang, Zipei Fan, Ryosuke Shibasaki, and Taisei Sato:
City-Scale Human Mobility Prediction Model by Integrating Gnss Trajectories and Sns Data Using Long Short-Term Memory, ISPRS Annals of the Photogrammetry, Remote Sensing and Spatial Information Sciences, Volume V-4-2020, 2020, pp.87-94, August 2020.
\bibitem{JIANG2005}
Quanjun Chen, Renhe Jiang, Chuang Yang, Zekun Cai, Zipei Fan, Kota Tsubouchi, Xuan Song, Ryosuke Shibasaki: 
DualSIN: Dual Sequential Interaction Network for Human Intentional Mobility Prediction, Proceedings of the 28th International Conference on Advances in Geographic Information Systems (SIGSPATIAL '20), pp.283–292, November 2020.
\bibitem{JIANG2006}
Xiaodan Shi, Xiaowei Shao, Guangming Wu, Haoran Zhang, Zhiling Guo, Renhe Jiang, Ryosuke Shibasaki: 
Social-DPF: Socially acceptable distribution prediction of futures, Proceedings of The Thirty-Fifth AAAI Conference on Artificial Intelligence (AAAI'21), February 2021.
\bibitem{HSM02}
S. Hayashi, M. Sugiyama, S. Matsushima: “Coordinate Descent Method for Log-linear Model on Posets,”  In Proceedings of IEEE International Conference on Data Science and Advanced Analytics (DSAA), pp. 99-108 (2020)
\bibitem{MB01}
S. Matsushima, M. Brbić: “Selective Sampling-based Scalable Sparse Subspace Clustering,” Advances in Neural Information Processing Systems (NeurIPS). pp. 12416-12425 (2019)
\bibitem{RSMZYV01}
P. Raman, S. Srinivasan, S. Matsushima, X. Zhang, H. Yun, S. V. N. Vishwanathan: “Scaling Multinomial Logistic Regression via Hybrid Parallelism,” ACM SIGKDD Conference on Knowledge Discovery and Data Mining (KDD), pp. 1460-1470 (2019)

\bibitem{SatoruNakamura401} 
中村覚, 佐治奈通子, 永崎研宣:
 TEI とIIIF をベースとしたオン/オフライン併合型史料研究支援システムの開発 - オスマン・トルコ語文書群を対象として,  じんもんこん2019論文集 2019 pp.293-300 2019.

\bibitem{SatoruNakamura402} 
小風尚樹, 中村覚, 永崎研宣:
 構造化記述された財務記録史料データの分析手法の開発:イギリスの船舶解体業を事例に,  じんもんこん2019論文集 2019 pp.183-190 2019.

\bibitem{SatoruNakamura403} 
Satoru Nakamura, Kazuhiro Okada, Kiyonori Nagasaki:
 An Attempt of Dissemination of TEI in a TEI-underdeveloped country: Activities of the SIG EAJ,  The 19th annual Conference and Members Meeting of the Text Encoding Initiative Consortium 2019.

\bibitem{SatoruNakamura404} 
Kazuhiro Okada, Satoru Nakamura, Kiyonori Nagasaki:
 An Encoding Strategic Proposal of “Ruby” Texts: Examples from Japanese Texts,  The 19th annual Conference and Members Meeting of the Text Encoding Initiative Consortium 2019.

\bibitem{SatoruNakamura405} 
Satoru Nakamura:
 Approach to develop Digital Collection for Small Organization considering Sustainability and Reusability with IIIF and Static File,  The 9th International Conference of Japanese Association for Digital Humanities pp.76-78 2019.

\end{査読付}

\begin{公開}{1}

\end{公開}

\begin{特許}{1}


\end{特許}

\begin{発表}{1}

\bibitem{01石川正俊11}
島田史也,村上健一,妹尾拓,石川正俊:6軸力センサを搭載したロボットハンドを用いた加振によるボトル内の液体判別,ロボティクス・メカトロニクス講演会(ROBOMECH2020)(オンライン)/ 講演論文集, 2A2-N16,2020/05.

\bibitem{01石川正俊12}
漆原昂,村上健一,妹尾拓,石川正俊:弾塑性変形制御を用いたヒューマンロボットインタラクション,ロボティクス・メカトロニクス講演会(ROBOMECH2020)(オンライン)/ 講演論文集, 2A2-C15, 2020/05.

\bibitem{02早川智彦14}
早川智彦,高原慧一,柯毓珊,石川正俊:半導体可視光レーザーによる加熱箇所の熱画像を利用した動的マーカー生成手法,一般社団法人レーザー学会学術講演会 第41回年次大会予稿集(オンライン),p.H03-20a-VIII-04:1,2021.1.

\bibitem{02早川智彦15}
久保田祐貴, 柯毓珊, 早川智彦, 石川正俊: 2種の材料を用いた着脱可能な赤外マーカーにおける撮像性能の検証, 映像情報メディア学会創立70周年記念大会 (オンライン),予稿集, 12E-2,2020/12.

\bibitem{02早川智彦16}
美間 亮太,久保田 祐貴,早川 智彦,石川 正俊:ベンハムのコマの無彩色化システムを用いた主観色の補償効果の評価,第21回計測自動制御学会システムインテグレーション部門講演会(オンライン)予稿集,pp. 1955-1957,2020/12.

\bibitem{02早川智彦17}
早川智彦,柯毓珊,望戸雄史,石川正俊:モーションブラー補償撮像手法を利用した走行型点検車両の照明要件―高速道路のトンネル覆工表面の撮影に向けて―,2020年度 第42回照明学会東京支部大会(オンライン)予稿集,pp. B-6:1-B-6:2,2020/12.

\bibitem{02早川智彦18}
早川智彦,柯毓珊,石川正俊:再帰性反射光の広がりによる空中結像を利用したディスプレイ空間拡張手法,第25回日本バーチャルリアリティ学会大会 (VRSJ2020) (オンライン)予稿集, 2B1-5: 1--2B1-5: 4,2020/09.

\bibitem{02早川智彦19}
早川智彦,望戸雄史,村上健一,石川 正俊:軌道材料の異常検出に向けた 鉄道巡航速度における高解像度画像撮影手法の提案,令和2年度土木学会全国大会 第75回年次学術講演会Web討論会(オンライン)/ WEB版年次学術講演会プログラム, VI892:1-VI892:3,2020/09.

\bibitem{02早川智彦20}
江崎ゆり子,望戸雄志,早川智彦,石川正俊:ガルバノミラーを用いた撮影角度の高速スイッチング,2020年 第45回光学シンポジウム(オンライン)講演論文集,pp.85-89,,2020/06.

\bibitem{02早川智彦21}
早川智彦, 栃岡陽麻里, 久保田祐貴, 美間亮太, 石川正俊:色と形に関する2種の錯視における知覚のフレームレート依存性, 映像表現・芸術科学フォーラム2021(映情学技報, vol. 45, no. 8, オンライン),pp.157-160,2021

\bibitem{02早川智彦22}
早川智彦,中根悠,蛭間友香,望戸雄史,石川正俊: シリコン単結晶立方体の回転動作を用いた移動時の熱画像撮影における空間分解能向上法, 2021年度精密工学会春季大会学術講演会講演論文集(オンライン),pp. 581-582,2021/03.

\bibitem{03黄守仁04}
黄守仁, 小山佳祐, 石川正俊, 山川雄司:両腕同期運動を利用した力覚提示による人間機械協調,ロボティクス・メカトロニクス講演会2020(ROBOMECH2020),講演論文集,2020.

\bibitem{04末石智大07}
末石智大,西薗良太,石川正俊:ベクター型レーザー投影系におけるM系列破線マーカーを用いたロバスト高速自己姿勢推定, 第21回計測自動制御学会システムインテグレーション部門講演会 (SI2020),講演会論文集, pp.1445-1448,2020/12.

\bibitem{04末石智大08}
松本明弓,末石智大,石川正俊: 注視点追従高解像度投影に向けた高速視線推定システム,第21回計測自動制御学会システムインテグレーション部門講演会 (SI2020),講演会論文集, pp.1886-1889,2020/12.

\bibitem{04末石智大09}
松村蒼一郎,末石智大,谷内田尚司,石川正俊:高速光学系制御を用いた頭部非拘束状態における眼球微振動検出手法,第21回計測自動制御学会システムインテグレーション部門講演会 (SI2020),講演会論文集, pp.1894-1898,2020/12.

\bibitem{04末石智大10}
末石智大,深山理,宮地力,山川雄司,石川正俊:ゴルフスイングのフォーム・幾何情報の逐次的高速投影システムの開発,第21回計測自動制御学会システムインテグレーション部門講演会 (SI2020),講演会論文集, pp.386-389,2020/12.

\bibitem{04末石智大11}
神宮亜良太,末石智大,谷内田尚司,石川正俊:遠隔虹彩認証に向けた高速光学系制御を用いた眼追従合焦撮影手法,第26回画像センシングシンポジウム (SSII2020),講演論文集, IS2-16,2020/6.


\bibitem{07平野正浩04}
川原 大宙,妹尾 拓,石井 抱,平野 正浩,岸 則政,石川 正俊:重畳車両の輪郭抽出に基づく高速トラッキング,第21回計測自動制御学会システムインテグレーション部門講演会 (SI2020),講演会論文集, pp.1457-1459,2020/12

\bibitem{08田畑智志02}
久一 空,野元 貴史,田畑 智志,渡辺 義浩:マルチパターン埋め込み型位相シフト法に基づく高速3次元計測の開発,第26回画像センシングシンポジウム,IS3-35,2020/06.

\bibitem{08田畑智志03}
野元 貴史, 田畑 智志, 渡辺 義浩: 偏光アレイカメラを用いた構造化光法による深度・法線の高速取得,第26回画像センシングシンポジウム,IS3-19,2020/06.

\bibitem{09角博文02}
Masayuki Uemura, Wang Ze,Hironari Takehara, Hirofumi Sumi,Hiroyuki Tashiro,Makito Haruta,Kiyotaka Sasagawa and Jun Ohta:Evaluation of mosaic multilayer interference filter attached on an image sensor for near-infrared color high-speed fundus camera,ITE Technical report,Vol.44, No 14,2020/01.

\bibitem{10宮地力02}
宮地 力:スポーツのトレーニングとロボット,日本ロボット学会誌,38-4, pp331-333, 2020.

\bibitem{10宮地力03}
宮地 力,中川康二:機械学習の基づくスポーツ用モーションキャプチャシステム概説,画像ラボ,pp26-33, 2021/02.

\bibitem{KM01} 上月正貴、松島慎「二変数間の相互作用を考慮した一般化加法モデルの効率的な学習」第22回情報論的学習理論ワークショップ、名古屋、2019年11月
\bibitem{HSM01} 林翔太、杉山麿人、松島慎「半順序構造上の対数線形モデルのための座標降下法」第22回情報論的学習理論ワークショップ、名古屋、2019年11月
\bibitem{NM01} 西本洋紀、松島慎「対数線形モデルを基とした生成的分類器と識別的分類器のロジスティック汎化誤差の収束の比較」第23回情報論的学習理論ワークショップ、オンライン、2020年11月
\bibitem{KM02} 上月正貴、松島慎「二変数間相互作用を考慮した一般化加法モデルとその効率的な学習」科研費シンポジウム機械学習・統計学・最適化の数理とAI技術への展開、オンライン、2020年12月
\bibitem{SatoruNakamura701} 
中村覚, 水野遊大, 稗方和夫, 成田健太郎:
 デジタル文化資料活用システムの設計手法 ―法帖研究支援の事例― ,  人工知能学会研究会資料 SIG-KST-039-02 pp.1-6 2020.

\bibitem{SatoruNakamura702} 
NAKAMURA Satoru:
 Development of Content Retrieval System of Scrapbook “Kunshujo” using IIIF and Deep Learning,  2019 IIIF Conference 2019.

\bibitem{SatoruNakamura703} 
NAKAMURA Satoru, NAGASAKI Kiyonori:
 IIIF Discovery in Japan,  2019 IIIF Conference 2019.

\bibitem{SatoruNakamura704} 
佐治奈通子, 中村覚:
 歴史学と情報学のより良い協働を目指して―オープンなDH支援ツールを用いたボスニアのカトリック修道院所蔵のオスマン・トルコ語文書群のデータ整理の一事例,  研究報告人文科学とコンピュータ(CH) 2019-CH-120(11) pp.1-7 2019.

\bibitem{SatoruNakamura705} 
Ayano Kokaze, Satoru Nakamura, Kiyonori Nagasaki, Naoki Kokaze:
 Enriching the Life Cycle of data: Supporting a project by DH approach,  International Society for Eighteenth-Century Studies Congress 2019 2019.

\end{発表}

\begin{特記}{1}

\bibitem{02早川智彦23}
MDPI Photonics, "Advances in 3OM: Opto-Mechatronics, Opto-Mechanics, and Optical Metrology", Special Issue Editors

\bibitem{02早川智彦23}
展示:早川智彦,柯毓珊,石川正俊:再帰性反射光の広がりによる空中結像を利用した光源拡張手法,第25回日本バーチャルリアリティ学会大会 (VRSJ2020) (オンライン)/Open Virtual Exhibition, A17-(3),2020/9.

\end{特記}

\begin{報道}{1}

\bibitem{02早川智彦24}
Itmedia BUILT,「時速100kmで覆工ンクリートの変状を検出するシステムが国交大臣賞」.
 
\bibitem{02早川智彦25}
国土交通省,【令和3年1月8日】,「第4回インフラメンテナンス大賞」表彰式に赤羽大臣が出席 .

\bibitem{02早川智彦26}
国土交通省,インフラメンテナンスの優れた取組や技術開発を表彰!,~第4回「インフラメンテナンス大賞」受賞者を決定~. 

\bibitem{02早川智彦27}
CGWORLD:「Digital Content EXPO 2020 Online Innovative Technologies 2020」.


\end{報道}
