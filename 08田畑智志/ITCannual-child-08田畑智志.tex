\subsection{研究報告(田畑 智志)}

高速な三次元計測および投影に関する研究

 物体の三次元的な形状・運動の高速取得や、三次元空間に対する情報の高速フィードバックは、現実空間とデジタル空間のインタラクションをはじめ、多くの応用の基盤となる技術である。そのため、高速三次元形状計測の高精度化や高解像化を異なるアプローチで改善するとともに、高速三次元スキャンを実現する小型デバイスの開発を行った。また、三次元空間に情報を提示するため高速焦点追従投影システムを開発した。高速三次元形状計測においては、これまでに提案している階層構造を持つセグメントパターンと三視点幾何拘束による1,000fpsの疎な形状計測手法に加えて照度差ステレオによる法線計測を組み合わせることで高精度化と高解像化を達成した。また、位相シフト法をベースとした高解像な形状計測手法に偏光アレイカメラを組み合わせた法線の同時計測手法にも取り組んでいる。一方で、2値の固定パターンであるセグメントパターンを単純な構成で投影することで計測装置の小型化にも取り組んでおり、開発したデバイスを用いて高速三次元形状計測と高速三次元運動計測を行うことで、手に持ったデバイスを用いて1,000fpsで物体の高速三次元スキャンを実現している。高速焦点追従投影システムでは、投影対象までの距離を高速に計測し、その情報をフィードバックして液体レンズと高速プロジェクタを制御するシステムを構築している。通常、一定距離のスクリーンに投影するプロジェクタは十分な光量を確保するために開口が広く被写界深度が浅い。そのため、等距離面内の移動に対しては高速プロジェクタによる追従投影が実現されていたものの、投影対象の距離の変化があると投影内容にピンボケが発生していた。これに対し、新たに液体レンズを用いた焦点距離制御を組み合わせることで、三次元的に移動する投影対象に対し距離に応じた内容をボケずに高速投影できる。このシステムを用いてVolume Slicing Displayといった情報提示も実現している。
