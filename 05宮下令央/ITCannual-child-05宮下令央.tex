\subsection{研究報告(宮下 令央)}

 本年度は通常の研究業務に加え、主に修士課程学生の指導補助と研究周知活動を行った。
研究業務では高速ビジョンチップを利用した小型のダイナミックプロジェクションマッピングシステムを開発し、研究発表を行った。さらに、別途PCを用意することなく高速ビジョンチップを幅広いシステムへ組み込むため、CPUとアナログ出力回路を搭載したスタンドアロンドーターボードを開発した。また、新たに開発したバイナリ法線画像特徴量と、距離と法線の高速同時計測システムの評価を行い、論文執筆を予定している。本年度は研究室の所属変更により公式には修士課程の学生は所属していないが、前年度本研究室に所属していた修士課程の学生の指導補助を継続している。本年度はダイナミックプロジェクションマッピングと人間の知覚特性を利用した変形錯視システムの開発と研究発表、被験者実験の実施を行った。さらに本研究をCGに応用したリアルタイムアニメーション生成アルゴリズムを開発した。いずれの結果も論文投稿に至っている。また、分光反射率特性と自然冷却特性を利用した質感計測アルゴリズムの改良と特許申請を行った。また、本年度はコロナによって多くの学会がオンラインとなり、プログラムが変更されたことによって講演の業務が増えている。企業向けに速度計測とダイナミックプロジェクションマッピングの講演を行い、学生向けに高速画像処理、学会参加者向けにダイナミックプロジェクションマッピングの講演を行った。さらに、研究プロジェクトの総括として3次元形状と質感計測の研究について講演し、研究周知に努めた。査読や学会の委員を拝命し、学会への貢献も引き続き行っている。


