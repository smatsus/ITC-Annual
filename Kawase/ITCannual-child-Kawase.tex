\subsection{野生動物ワイヤレスセンサネットワーク実証実験基盤構築に向けた研究(川瀬)}
\subsubsection{概要}
インフラ基盤のない野生環境下での運用を想定した野生動物装着型ワイヤレスセンサネットワーク(WSN)の開発を念頭に、放牧下の家畜動物による効率的な評価実験基盤の構築を目指している。2020年度は国内の放牧場に協力を依頼し、実験基盤の構築・設置、現地での評価実験実施を目指して進めてきたが、コロナ禍により中断している。また野生動物装着型WSNに特化したデータ共有手法について検討を進めている。
\subsubsection{内容}
放牧下の家畜動物たちがモバイルセンサを持ち歩き、単独行動時に取得したデータを集団行動時に共有する。そして最終的にシンクノード付近に滞在する個体からデータを回収し、モバイル通信を介して遠隔地でデータを蓄積・リアルタイムでの分析を一体的に試みることを目指す。既に、野生動物装着型WSNを見据えた動物装着モバイルセンサノードの開発・実験、モバイル通信による広域データ収集基盤の試運転を行ってきた。そこで、本研究では放牧場及び放牧家畜の協力のもと、一体的な評価実験基盤の構築及び評価実験を実施する。
\subsubsection  {成果報告}
本研究は、2020年度国立情報学研究所公募型共同研究として進められた。北海道安平町や岩手県久慈市などの肉牛放牧場らと実験基盤の構築・設置、現地での評価実験実施の交渉を行っていたが、コロナ禍により中断している。また、野生動物装着型WSNでは、いつ・どの組み合わせで発生するかわからない野生動物同士の遭遇を考慮したうえで、無線通信での効率的かつ精確なデータ共有手法が必要になる。そこで、このデータ共有手法について検討を進めている。
