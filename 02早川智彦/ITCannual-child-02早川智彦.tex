\subsection{研究報告(早川 智彦)}

 2020年度は主に1.高速画像処理技術によるモーションブラー補償と2.人間の知覚情報の定量化、3.ディスプレイの拡張表現技術の研究を実施した。全体を通した研究成果として、6件の表彰を受け、1件の雑誌論文(査読付)と5件の雑誌以外の査読付き論文を投稿し、10件の発表を行った。

1.高速画像処理技術によるモーションブラー補償 インフラ点検に関する撮像技術の開発を行い、回転キューブを用いる光軸制御法を提案することで、よりサンプリングレートの高い撮像手法の検討を行った。更に赤外領域を撮像することによって、コンクリート表面だけでなく内部変状も移動しながら計測する研究や、半導体可視光レーザーによる動的マーカーの開発を行っている。

2.人間の知覚情報の定量化 高速カメラと高速プロジェクタを用い、被験者実験により24ms以上の遅延がパフォーマンス低下を引き起こすことを発表した。錯視の研究では人間の眼球運動と視線位置に基づき、映像に生じる錯視をリアルタイムに補償するシステムおよびアルゴリズムを提案した。また、錯視における知覚のフレームレート依存性を調査し映像における錯視表現に必要な値を明らかにした。これらの研究により、インタラクティブな没入型デバイスにおける設計やリアリティーの高い映像表現の指標を示した。

3.ディスプレイの拡張表現 特定速度で移動している人だけに伝達可能な二次元情報提示システム「Bilateral Motion Display」の開発を行った。観測するユーザの身体や視線の動きの速度・方向に応じて、それぞれに異なる映像を知覚させる指向性多義ディスプレイを実現した。高速に投影された画像の一部成分が、残像として重なり合う効果を利用している。また、再帰性反射材を用いた反射光の広がりによる空中結像を利用したディスプレイ拡張手法の発表を行った。これらの研究はセキュリティ分野や広告、標識、エンターテインメントへの活用が期待される。
