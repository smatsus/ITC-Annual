\subsection{研究報告(末石 智大)}

 高速画像処理および高速光学系制御を用いた、動的検査技術とヒューマンインターフェースに関する研究を行った。動的検査技術は、実世界に存在する複雑な現象を適切にデータ化・活用する技術であり、泳ぎ回るメダカや人間の眼の虹彩、眼球微振動などを対象として実施した。静止状態における検査技術は数多くあるが、時間効率が低い・被写体に負荷がかかるなど、動的状態への検査技術の発展の期待は大きいと考えられる。歩いている人などを含め静止状態を作り出せない状況などへの発展を目指し、回転ミラーや液体可変焦点レンズなどの光学素子を高速に制御し、運動対象の高解像度撮影を達成することで運動物体への検査のための基礎技術開発を進めている。水族館や養殖事業を意図したメダカの健康状態計測、歩いている人への虹彩認証技術、頭部非拘束状態の人への眼球微小変化計測技術などの達成に向けた基礎的成果を実現した。ヒューマンインターフェースに関しては、ダイナミックプロジェクションマッピングやアイトラッキング、高速自己位置推定に加え、球技やスイング動作を含むスポーツ応用に結び付く内容も実施した。人間の挙動に関連した情報をデータ化するだけでなく、高速な情報提示も行うことで人間に役立つ形で活用するところまで達成する技術である。眼や身体動作など特に高速な挙動に着目して取り組んだが、特にスポーツで扱うボールやゴルフクラブなどの更に高速な運動に対しても高速にセンシングを行うマーカー技術・画像処理技術を生み出し、その動きに対応した高速投影技術を創り出した。