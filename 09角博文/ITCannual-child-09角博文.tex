\subsection{研究報告(角 博文)}

 マルチ近赤外線カメラを用いた全く眩しくない動画撮影可能な眼底カメラを奈良先端科学技術大学院大学 太田研究室と共同で開発した。複数毎得られる動画像データの画像にラベリングを行い良好な画像だけ取得し積分させることで画像の高SN化を実現した。(東北大鏡先生共同)微細画素で複数の近赤外線領域のバンドパスを実現できるプロセスを開発した。フルHDのイメージセンサにベイヤー配列のバンドパス構造を3um画素に適応し、3種類の近赤外線波長をそれぞれの画素の信号として得る事ができるカメラシステムである。このプロセスで特徴的なのは、複数のバンドパスフィルタを構成する多層膜の下地半分は画素領域で全て層として繋がっている。約40nm弱の中心部の膜厚だけ数十nm変える事で目的のバンドパス特性を得る。さらにその上上層膜も全て切れ目なしにつながっている構造である。バンドパス波長の決定付ける膜厚は100nm以下で、隣り合った画素では数十nm以下の膜厚差でパターニングされる。この構造を応用して3種類のフィルタを2種類のパターニングで可能となるプロセスを開発した。断面TEMによる構造解析も行い目的の膜厚で3種類の近赤外線バンドパスフィルタが形成できている事を確認した。このマルチ近赤外線カメラで眼底観察を行い静脈及び動脈の状態の観察が可能となった。これらの観察で動脈硬化、高血圧などの目の疾患以外の健康状態も把握できるカメラである。現在大阪大学大学病院とも共同で開発を進めている。

